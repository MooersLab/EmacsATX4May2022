 % !TEX TS-program = pdflatex --shell-escape
%\documentclass[handout]{beamer} %slides+notes only
%\documentclass[10pt,t]{beamer} %slides only
% The laulatex is being updated. The following RequirePackage line is needed for now.
%
% Under Menu, select compiler 2017 (Legacy) to get animate to make a pdf with a working animation.
% The 2019 compiler does not work as of 18 September 2020.
%
% Hold down the wheel or middle mouse button pause the animation.
% Add the control keyword to see the controls.
%
\documentclass[aspectratio=169]{beamer}
\RequirePackage{luatex85}
\usepackage{animate}
\usepackage{pgfpages}

\usepackage{moresize}
\usepackage{multicol}
\usetheme{default}
\beamertemplatenavigationsymbolsempty
\hypersetup{pdfpagemode=UseNone} % don't show bookmarks on initial view
%tables
\usepackage{booktabs}% http://ctan.org/pkg/booktabs
% font
\usepackage{amsmath}
\usepackage{amssymb}
\usepackage{minted}
\usepackage{fix-cm}
\usepackage{fontspec}
\usepackage{gensymb}
\usepackage{tabularx,booktabs}
\usepackage[T1]{fontenc}
\usepackage{textcomp}
\usepackage{upquote}
\AtBeginDocument{%
\def\PYZsq{\textquotesingle}%
}


\usepackage{enumitem}

\newminted{python}{fontsize=\normalsize, 
                   linenos=false,
                   numbersep=8pt,
                   gobble=4,
                   breaklines,
                   frame=lines,
                   bgcolor=bgpython,
                   framesep=3mm}

\newminted{bash}{fontsize=\largesize, 
                   linenos=false,
                   numbersep=8pt,
                   breaklines,
                   gobble=4,
                   frame=lines,
                   bgcolor=bgbash,
                   framesep=3mm}

\newminted{elisp}{fontsize=\largesize, 
                   linenos=false,
                   numbersep=8pt,
                   breaklines,
                   gobble=4,
                   frame=lines,
                   bgcolor=bgelisp,
                   framesep=3mm}
                   
\definecolor{bgpython}{rgb}{0.95,0.95,0.95}
\definecolor{bgbash}{rgb}{0.95,0.85,0.75}
\definecolor{bgelisp}{rgb}{0.95,0.85,0.75}

%\setbeameroption{show only notes}
\setsansfont{TeX Gyre Heros}
\setbeamerfont{note page}{family*=pplx,size=\footnotesize} % Palatino for notes
% "TeX Gyre Heros can be used as a replacement for Helvetica"
% In Unix, unzip the following into ~/.fonts
% In Mac, unzip it, double-click the .otf files, and install using "FontBook"
%   http://www.gust.org.pl/projects/e-foundry/tex-gyre/heros/qhv2.004otf.zip

% Center the title and increase its size
\setbeamertemplate{frametitle}[default][center]
\setbeamerfont{frametitle}{size=\huge}


% named colors
\definecolor{offwhite}{RGB}{249,242,255}
\definecolor{foreground}{RGB}{25,25,25}
\definecolor{background}{RGB}{255,255,255}
\definecolor{title}{RGB}{100,0,0}
\definecolor{gray}{RGB}{155,155,155}
\definecolor{subtitle}{RGB}{50,0,0}
\definecolor{hilight}{RGB}{102,255,204}
\definecolor{vhilight}{RGB}{255,111,207}
\definecolor{lolight}{RGB}{155,155,155}
%\definecolor{green}{RGB}{125,250,125}

% use those colors
\setbeamercolor{titlelike}{fg=title}
\setbeamercolor{subtitle}{fg=subtitle}
\setbeamercolor{institute}{fg=gray}
\setbeamercolor{normal text}{fg=foreground,bg=background}
\setbeamercolor{item}{fg=foreground} % color of bullets
\setbeamercolor{subitem}{fg=gray}
\setbeamercolor{itemize/enumerate subbody}{fg=gray}
\setbeamertemplate{itemize subitem}{{\textendash}}
\setbeamerfont{itemize/enumerate subbody}{size=\footnotesize}
\setbeamerfont{itemize/enumerate subitem}{size=\footnotesize}

% page number
\setbeamertemplate{footline}{%
    \raisebox{5pt}{\makebox[\paperwidth]{\hfill\makebox[20pt]{\color{gray}
          \scriptsize\insertframenumber}}}\hspace*{5pt}}

% add a bit of space at the top of the notes page
\addtobeamertemplate{note page}{\setlength{\parskip}{12pt}}

% a few macros
\newcommand{\bi}{\begin{itemize}}
\newcommand{\ei}{\end{itemize}}
\newcommand{\ig}{\includegraphics}
\newcommand{\subt}[1]{{\footnotesize \color{subtitle} {#1}}}

\usepackage{latexsym} % for squares for the check-list environment
\newenvironment{checklist}{%
  \begin{list}{}{}% whatever you want the list to be
  \let\olditem\item
  \renewcommand\item{\olditem[$\Box$] }
}{%
  \end{list}
}


% title info
\title{Functional elisp from the beginning} 
\author{\textbf{\large{Blaine Mooers, PhD} \\ blaine-mooers@ouhsc.edu \\ 405-271-8300}}
\institute{{Department of Biochemistry \& Molecular Biology}\\[2pt]{University of Oklahoma Health Sciences Center, Oklahoma City} }
% to hide auto date,use \date{}
\date{Austin Emacs Meetup\\ Austin, Texas\\ Virtual Meeting\\ 4 May 2022}

\begin{document}

% title slide
{
\setbeamertemplate{footline}{} % no page number here
\frame{
  \titlepage
 \note{
Hi, I am Blaine Mooers.
I am an associate professor of Biochemistry and Molecular Biology at the University of Oklahoma Health Sciences Center in Oklahoma City, Oklahoma.
I am protein crystallographer or a so-called structural biology.
I work at the insection of chemistry, biology, physics, and computing.

I attended the meeting in February and volunteered to give a talk on functional programming with Emacs lisp to force myself to learn some elisp and a little about functional programming.
I first started popping open Emacs several years ago for the purpose of making a yasnippet snippet library for doing literate programming with the molecular graphics program PyMOL.
I was making this library available for 20 leading text editors.
At the time, my favorite text editor for getting things done fast was TextMate.
In the process, I learned some basic Vim. 
A year ago, I decided to give Emacs a try.
Several months ago, I ditched evil, mode and I am trying to master the Emacs Key bindings.

I poked around in Spacemacs, Doom, Prelude and SciMax long enough to decide to go back to the basics in Gnu Emacs.
In the 
I have built and rebuilt my configuration file several times.
I discovered that the chemacs2, the Emacs profiler switcher, is great for retaining a broken configuration while rebuilding a second one.
}
} }

\section{Introduction}
\subsection{Who am I}

\section{Workflow in the Mooers Lab}
\begin{frame}
\frametitle{Workflow in the Mooers Lab}
\begin{center}
    \includegraphics[width=0.79\textwidth, angle=0]{./Figures/workflow}
\end{center}
\end{frame}
\note{
My lab uses X-ray crystallography to determine the atomic structures of protiens and nucleic acids important in human medicine. 
This figure shows our workflow
We grow crystals of purified proteins and RNA like the crystal of RNA in the upper left. 
We shoot the crystal with X-rays as we slowly rotate the crystal on a X-ray diffraction instrument, as shown in the top center.
The image in the upper right shows thousands of diffraction spots from a two degree rotation.
We rotate the crystal 180 degrees to collect a full dataset with 90 images. 

%We integrate, merge, and scale the diffraction spots. and then 
We take an inverse Fourier transform of the diffraction data to generate an electron density map like the one shown in the bottom, center panel.
We fit molecular models into the maps on computers and refine the model using maximum likelihood methods to improve its fit. 
The final model is then used in a molecular graphics program like PyMOL to make images for publication like the one in the lower right. 
}


%%%%%%%%%%%%%%%%%%%%%%%%%%%%%%%%%%%%%%%%%%%%   3  %%%%%%%%%%%%%%%%%%%%%%%%%%%%%%%%%%%%%%%%%%%%    
% Slide 3
\section{Cover images made with PyMOL}
\begin{frame}
\frametitle{Cover images made with PyMOL}

\begin{center}
    \includegraphics[width=0.79\textwidth, angle=0]{./Figures/trimmedCoverFigs}
\end{center}

\end{frame}
\note{
PyMOL has over 100,000 users and is the most popular program for making images of protein structures for publication. 
PyMOL has about 500 commands and 600 parameters that enable the user to exert exquisite control over the appearance of the image.
Shown are four examples of images made with PyMOL on the covers of several top journals.

}



%%%%%%%%%%%%%%%%%%%%%%%% 4  %%%%%%%%%%%%%%%%%%%%%%%%%%%%%%%%%%%    
% Slide 4
\section{PyMOL GUI}
\begin{frame}
%\frametitle{PyMOL GUI}

\begin{center}
    \includegraphics[width=0.72\textwidth, angle=0]{./Figures/retkinase}
\end{center}

\end{frame}
\note{
This the GUI for PyMOL. 
Python can be run with pulldown menus, commands at the PyMOL prompt, or with scripts.
The PyMOL prompt is actually an interactive Python prompt.
PyMOL is written in C for speed, but wrapped with Python for ease of extensibility.
There are about 300 plugins that users have made for PyMOL.
}


%%%%%%%%%%%%%%%%%%%%%%%% 5  %%%%%%%%%%%%%%%%%%%%%%%%%%%%%%%%%%%    
% Slide 5
\section{Customized molecular representations}
\begin{frame}
\frametitle{Customized molecular representations}

\begin{center}
    \includegraphics[width=0.72\textwidth, angle=0]{./Figures/cartoonsHairpin}
\end{center}

\end{frame}
\note{
These are four representations of a 27-nucleotide RNA hairpin.
The default cartoon from PyMOL is on the left.
The three representations to the right were made using Python scripts that access the PyMOL API.
Their code is in our snippet library.
These examples demonstarte that PyMOL can be extended.
}

%%%%%%%%%%%%%%%%%%%%%%%% 5  %%%%%%%%%%%%%%%%%%%%%%%%%%%%%%%%%%%    
% Slide 5
\section{Missing molecular representations}
\begin{frame}
\frametitle{Missing molecular representations}

\begin{center}
    \includegraphics[width=0.65\textwidth, angle=0]{./Figures/hairpinAOs}
\end{center}
\url{https://github.com/MooersLab/pymolshortcuts}\\
\url{https://github.com/MooersLab/pymolsnips}
\end{frame}
\note{
These are four representations of a 27-nucleotide RNA hairpin.
The default cartoon from PyMOL is on the left.
The three representations to the right were made using Python scripts that access the PyMOL API.
Their code is in our snippet library.
These examples demonstarte that PyMOL can be extended.
}

%%%%%%%%%%%%%%%%%%%%%%%% 7 %%%%%%%%%%%%%%%%%%%%%%%%%%%%%%%%%%%    
% Slide 7
\section{org-yas pull down}
\begin{frame}
\frametitle{orgpymolpysnips}

\begin{center}
    \includegraphics[width=0.8\textwidth, angle=0]{./Figures/org-yas.png}
\end{center}
\url{https://github.com/MooersLab/orgpymolpysnips}
\end{frame}
\note{
The orgpymolpysnips librarys has 260 snippet divided into 20 categories for use in org-mode docuements.
These categories serve as groups in yasnippets that appear as submenus in the org-mode pulldown.
They are prepended with pymolpy- to distinguish them from non-PyMOL related groups of snippets.
}

%%%%%%%%%%%%%%%%%%%%%%%% 8 %%%%%%%%%%%%%%%%%%%%%%%%%%%%%%%%%%%    
% Slide 8
\section{Code block}
\begin{frame}
\frametitle{Code block to make one image}

\begin{center}
    \includegraphics[width=0.8\textwidth, angle=0]{./Figures/combined1.png}
\end{center}
\url{https://emacsconf.org/2021/talks/molecular/}
\end{frame}
\note{
Each snippet is flanked by the org-mode source block begin and end lines.
The top line specifies the kernel that calls PyMOL's Python API to run the code.
You place cursor on the top line or in the code block and enter C-c C-c. 
}


\section{Result block}
\begin{frame}
%\frametitle{Code block to make one image}

\begin{center}
    \includegraphics[width=0.80\textwidth, angle=0]{./Figures/pymol1out.png}
\end{center}

\end{frame}
\note{
I sent the output to a drawer.
This enables seeing the image in the file but it needs to be modest in size else scrolling will be difficult.
It may be best to close most of the code blocks and result blocks most of the time to avoid lags.
}


% %%%%%%%%%%%%%%%%%%%  2  %%%%%%%%%%%%%%%%%%%%%%%    
% % Slide 2
\begin{frame}
\frametitle{Levels of expertise}
 \begin{center}
\includegraphics[scale=0.79]{./Figures/Dreyfus}
\end{center}
\small{Dreyfus, S. E. and Dreyfus, H. L. (1980) A five-stage model of the mental activities involved in directed skill acquisition. DTIC Document.}
\end{frame}
\note{
Two Dreyfus brothers at UC Berkely proposed a model of expertise that had five levels in a report to the Department of Defense in the later 1970s. 
I find this model to be very useful. 
The report describes the features the different levels of expertise.
I would put XXX XXX, the speaker at the last meeting, was in the proficient to expert category.
I put myself in the advanced beginner category.
One way to move from beginning an advanced beginner to a competant user is to give talks about topics that you yet to have mastered.

I will assume no prior knowledge of emacs lisp.
I spent several years editing my config file to add packages without understanidng the code in my config file.
I got away with installing hundreds of packages without undering any elisp.
This talk is for such a person. 
I will cover just enough elisp that you will be able to follow along the introduciton to the functional features of elisp.
Of course, I do not have time to cover all of elisp.
I will assume that you have done at least a modest amount of programming.
}


% %%%%%%%%%%%%%%%%%%%  3  %%%%%%%%%%%%%%%%%%%%%%%    
% % Slide 2
\begin{frame}
 \begin{center}
\includegraphics[scale=0.2]{./Figures/FunctionalProgrammingSimplified}
\end{center}
\end{frame}
\note{
}


%%%%%%%%%%%%%%%%%%%%%%   2   %%%%%%%%%%%%%%%%%%%%%%%%%%%%%    
% Slide 2
\begin{frame}
\frametitle{Functional Programmming}
\Large{
\begin{itemize}[font=$\bullet$\scshape\bfseries]
\item A programming paradigm characterized by the use of \textbf{mathematical functions} and the avoidance of \textbf{side effects}.
\vspace{5mm}
\item A programming style that uses only \textbf{pure functions} without \textbf{side effects}.
\end{itemize}
}
\end{frame}
\note{
Functional programming has a 60+ year history.

These are the traditional definitions of functional programmming.
Eric Normand discuss why this traditional definition is programmatic. }

%%%%%%%%%%%%%%%%%%%%%%   2   %%%%%%%%%%%%%%%%%%%%%%%%%%%%%    
% Slide 2
\begin{frame}
\frametitle{Features of functional programmming}
\Large{
\begin{itemize}[font=$\bullet$\scshape\bfseries]
\item Variables are immutable.
\item Recursion instead of loops.
\item Pure functions without side effects.
\item Higher order functions 
(functions that take functions as arguments) 
\item Division of programs into actions, calculations, and data (functional thinking).
\end{itemize}
}
\end{frame}
\note{
Functional programming has a 60+ year history.

These are the traditional definitions of functional programmming.
Eric Normand discuss why this traditional definition is programmatic. }




\begin{frame}
\frametitle{Why learn functional programming \\(at least programming with functional thinking)?}
\Large{
\begin{itemize}[font=$\bullet$\scshape\bfseries]
\item Cleaner code that is easier to debug and maintain
\item More elegant 
\item Sometimes faster code
\item Makes you a better programmer
\vspace{5mm}
\end{itemize}
}
\end{frame}
\note{
Functional programming has a 60+ year history.

These are the traditional definitions of functional programmming.
Eric Normand discuss why this traditional definition is programmatic. }



% %%%%%%%%%%%%%%%%%%%  2  %%%%%%%%%%%%%%%%%%%%%%%    
% % Slide 2
\begin{frame}
\frametitle{Two topics}
\Large{
\vspace{5mm}
\begin{itemize}[font=$\bullet$\scshape\bfseries]
\item Ways to execute elisp code 
\vspace{5mm}
\item Functional features of elisp
\end{itemize}
}
\end{frame}
\note{
The cover of this book by Alvin Alexander was to good to be pass over.
It expresses the angst about functional programming that is similar to 
the angst that procedural-only programmers have when the topic of object-oriented 
programming is raised.
}


\section{Methods of running elisp}

%%%%%%%%%%%%%%%%%%%%%%   2   %%%%%%%%%%%%%%%%%%%%%%%%%%%%%    
% Slide 2
\begin{frame}
\frametitle{Methods of running elisp}
\Large{
% requires the enumitem package
\begin{itemize}[font=$\bullet$\scshape\bfseries]
    \item elisp file (.el)\\
    (documentation 'main) ; C-x C-e at space to right of `)'.\\
    (documentation 'main) ; C-M-x inside of the parentheses.
    \vspace{4mm}    
    \item scratch buffer ; C-j
    \vspace{4mm}
    \item M-x eval-buffer ; evaluate whole buffer
    \item M-x eval-region ; evaluate region
\end{itemize}
}
\end{frame}
\note{
LISP is a contraction of List Processing.
Lists are the 
ElISP Here are some examples of images made in PyMOL on the covers of leading journals. 
}

%%%%%%%%%%%%%%%%%%%%%%   2   %%%%%%%%%%%%%%%%%%%%%%%%%%%%%  


% Slide 2
\begin{frame}
\frametitle{Elisp REPLs (3 options inside Emacs)}
\Large{
% requires the enumitem package
\begin{itemize}[font=$\bullet$\scshape\bfseries]
    \item M-: ; REPL in mini buffer
    \vspace{4mm}    
    \item M-x ielm ; eval one expression at a time\\
       (defalias 'erepl 'ielm) \\
    M-x erepl
    \vspace{4mm}
     \item M-x eshell
\end{itemize}}
\end{frame}
\note{
The first option is to use ielm, which is in MELPA.
This name is not easy to remember.
I defined an alias to erepl.

Outside of Emacs, you define an alias. 

https://github.com/kanaka/mal
}


\defverbatim[colored]\exampleCodeA{
\large{
\begin{elispcode}
    alias erepl="rlwarp emacs --batch --eval \"(progn (require 'cl) (loop (print (eval (read)))))\""
\end{elispcode}
}
}

% Slide 2
\begin{frame}
\frametitle{elisp REPL ouside of Emacs}
\Large{
\exampleCodeA
Source: \url{https://www.reddit.com/r/emacs/comments/58ciar/question_can_elis_be_run_outside_of_emacs/}
}
\end{frame}
\note{
The first option is to use ielm, which is in MELPA.
This name is not easy to remember.
I defined an alias to erepl.

Outside of Emacs, you define an alias. 

https://github.com/kanaka/mal
}




\defverbatim[colored]\exampleCodeC{
\large{
\begin{pythoncode}
    #!emacs --script
    (defun main ()
      "Print the version of Emacs and print arguments as a list. 
      Example: chmod +x test5.el && ./test5.el dog cat
      Modified test4.el script by Dr. John Kitchin found here: 
      https://kitchingroup.cheme.cmu.edu/blog
      /2014/08/06/Writing-scripts-in-Emacs-lisp/"
    (print (version))
    (print (format "Called with %s" command-line-args))
    (print (format "You did it! You passed in %s!" command-line-args-left))
    (print (format "This is the documentation string for main():
      %s" (documentation 'main))) )
    (when (member "-scriptload" command-line-args) (main))
\end{pythoncode}
}
}
% Slide 24
\begin{frame}
\frametitle{Elisp script}
\exampleCodeC
\end{frame}





\defverbatim[colored]\exampleCodeC{
\large{
\begin{pythoncode}
    #+BEGIN_SRC emacs-lisp :results value scalar
    (*40 1000 1000 1000 1000 1000 1000 1000)
    #+END_SRC
    #+RESULTS:
    : 40000000000000000000
\end{pythoncode}
}
}
% Slide 24
\begin{frame}
\frametitle{Code block with output in the RESULTS drawer}
Place cursor inside code block and enter C-c C-c to run the code.
\exampleCodeC
Note that the ``:results value scalar'' was needed with emacs-lisp.
In the org-babel configuration, emacs-lisp has to be in the list of languages.
\end{frame}


\defverbatim[colored]\exampleCodeC{
\large{
\begin{pythoncode}
    (setq show-paren-delay 0)
    (show-paren-mode t)
\end{pythoncode}
}
}
% Slide 24
\begin{frame}
\frametitle{Config for matching parentheses}
\exampleCodeC
 \textquotesingle{}
\end{frame}

